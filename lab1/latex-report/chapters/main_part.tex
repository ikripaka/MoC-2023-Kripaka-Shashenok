\begin{filecontents*}{stochastic_table.csv}
#,M0,M1,M2,M3,M4,M5,M6,M7,M8,M9,M10,M11,M12,M13,M14,M15,M16,M17,M18,M19
C0,0,0,0,0,0,0,0,0,0,0,0,0,0,0,0,0,0,1,0,0
C1,0,0,1,0,0,0,0,0,0,0,0,0,0,0,0,0,0,0,0,0
C2,0,1,0,0,0,0,0,0,0,0,0,0,0,0,0,0,0,0,0,0
C3,0,1,0,0,0,0,0,0,0,0,0,0,0,0,0,0,0,0,0,0
C4,0,1,0,0,0,0,0,0,0,0,0,0,0,0,0,0,0,0,0,0
C5,0,0,0,0,0,0,0,0,0,0,0,0,0,1,0,0,0,0,0,0
C6,0,0,1,0,0,0,0,0,0,0,0,0,0,0,0,0,0,0,0,0
C7,0,0,0,1,0,0,0,0,0,0,0,0,0,0,0,0,0,0,0,0
C8,0,0,0,1,0,0,0,0,0,0,0,0,0,0,0,0,0,0,0,0
C9,0,0,0,0,0,0,1,0,0,0,0,0,0,0,0,0,0,0,0,0
C10,0,0,1,0,0,0,0,0,0,0,0,0,0,0,0,0,0,0,0,0
C11,1,0,0,0,0,0,0,0,0,0,0,0,0,0,0,0,0,0,0,0
C12,1,0,0,0,0,0,0,0,0,0,0,0,0,0,0,0,0,0,0,0
C13,0,0,1,0,0,0,0,0,0,0,0,0,0,0,0,0,0,0,0,0
C14,0,0,0,0,0,0,0,0,0,0,0,0,0,0,0,0,0,0,0,1
C15,0,0,0,1,0,0,0,0,0,0,0,0,0,0,0,0,0,0,0,0
C16,1,0,0,0,0,0,0,0,0,0,0,0,0,0,0,0,0,0,0,0
C17,0,0,1,0,0,0,0,0,0,0,0,0,0,0,0,0,0,0,0,0
C18,0,0,0,0,0,0,0,0,0,0,0,0,0,1,0,0,0,0,0,0
C19,0,0,0,1,0,0,0,0,0,0,0,0,0,0,0,0,0,0,0,0
\end{filecontents*}

\section{Мета практикуму}
Практично ознайомитися із тестами перевірки чисел на простоту, методами генерації ключів для асиметричної криптосистеми типу RSA, протоколом розсилання ключів та безпосередньо реалізувати їх.

\subsection{Постановка задачі та варіант}
\begin{tabularx}{\textwidth}{X|X}
	\textbf{Треба виконати} & \textbf{Зроблено} \\
	Описати побудову алгоритму \checkmark\\
	Порахувати таблицю ймовірностей $P(\textit{M}|\textit{C})$ \checkmark\\
	Показати детермінитичну та стохастичну функції у вигляді таблиць & \checkmark\\
	Порахувати середні витрати для вирішуючих функцій & \checkmark\\
\end{tabularx}



\section{Хід роботи/Опис труднощів}
???


\section{Результати дослідження}
У ході роботи було визначено, що детерміністична та стохастична вийшли однаковими.
Саме це не впливає ні на що, просто саме із такими початковими даними вийшло ось так.

\subsection{Опис алгоритму}



\subsection{Таблиця ймовірностей}



\subsection{Детерміністична та стохастична матриці}
Наведімо детерміністичну вирішуючу функцію у форматі відображення (ШТ \shortrightarrow ВТ), де середнє значення втрат становить -- $0.8376049$.

\begin{tabularx}{\textwidth}{|XX|}
	\textbf{ШТ} & \textbf{ВТ} \\
	0 -> & 17 \\
	1 -> & 2 \\
	2 -> & 1 \\
	3 -> & 1 \\
	4 -> & 1 \\
	5 -> & 13 \\
	6 -> & 2 \\
	7 -> & 3 \\
	8 -> & 3 \\
	9 -> & 6 \\
	10 -> & 2 \\
	11 -> & 0 \\
	12 -> & 0 \\
	13 -> & 2 \\
	14 -> & 19 \\
	15 -> & 3 \\
	16 -> & 0 \\
	17 -> & 2 \\
	18 -> & 13 \\
	19 -> & 3 \\
\end{tabularx}

Також ось і стохастична матриця, де середнє значення втрат становить -- $0.936809$.

\csvautotabular{stochastic_table.csv}

\subsection{Середні витрати для вирішуючих функцій}

\section{Висновки}
За допомогою реалізації практикуму ''Вивчення криптосистеми RSA та алгоритму електронного
підпису'' дізнався на практиці, як генеруться параметри для асиметричних криптосистем та як реалізовувати їх. 

Рекомендую для алгоритмів знаходження псевдопростих чисел одразу додавати алгоритм пробних ділень. Він значно скоротить час перебору чисел.

		

	




